\chapter{Hypotheses Development}

\section{Identified Gaps in the Literature}
Limited integration of unsupervised clustering and macroeconomic factors in volatility modeling.
Lack of robust frameworks for regime detection using machine learning.
Insufficient focus on predictive accuracy in existing models.

\section{Research Objectives}
Develop a framework integrating unsupervised clustering and macroeconomic factors.
Evaluate its effectiveness in identifying and predicting volatility regimes.
Assess its predictive accuracy compared to traditional models.

\section{Hypotheses}

\textbf{H1:} Integration of Unsupervised Clustering and Macroeconomic Factors
A framework combining unsupervised clustering and macroeconomic factors will more accurately identify volatility regimes than models using financial data alone.
Rationale: Macroeconomic factors (e.g., interest rates, GDP growth) influence market volatility, as shown by Diebold \& Yilmaz (2012) \cite{diebold_yilmaz_2012} and Giglio \& Kelly (2023).

\textbf{H2:} Effectiveness of Unsupervised Clustering
Unsupervised clustering techniques (e.g., k-means, hierarchical clustering) will outperform traditional methods (e.g., GARCH) in detecting volatility regimes.
Rationale: Clustering captures complex, non-linear patterns missed by traditional models, as demonstrated by Lopez de Prado (2018) and Bucci & Benoit (2023).

\textbf{H3:} Predictive Accuracy
The proposed framework will exhibit higher predictive accuracy in forecasting volatility regimes compared to traditional models.
Rationale: Combining financial data and macroeconomic indicators improves forecasting, as highlighted by Gu et al. (2020) and Chen et al. (2022).

\textbf{H4:} Macroeconomic Factors as Drivers of Regime Shifts
Macroeconomic factors will significantly drive regime shifts in volatility, improving the detection of abrupt changes.
Rationale: Macroeconomic shocks trigger sudden volatility shifts, as evidenced by Hamilton (1989) and Giglio \& Kelly (2023).

\textbf{H5:} Robustness Across Market Conditions
The framework will robustly identify and predict volatility regimes across different market conditions (e.g., bull markets, crises).
Rationale: A robust framework should perform consistently, as emphasized by Lopez de Prado (2018) \cite{lopez_de_prado_2018} and Bianchi et al. (2017).


Alignment with Research Objectives
H1 and H2: Align with developing the framework.
H3: Aligns with evaluating predictive accuracy.
H4 and H5: Align with assessing robustness and macroeconomic impact.
