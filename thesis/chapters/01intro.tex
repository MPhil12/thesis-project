\chapter{Introduction}
Here you should put a short introduction to your chapter. What is covered? In how much detail? Imagine you were coming back to this in 10 years time and wanted to find that one key equation, this part of the chapter should orient the reader to help find that information.

\section{Context and Problem Statement} \label{sec:sections}

Volatility is a fundamental characteristic of financial markets, representing the degree of variation in asset prices over time. Periods of heightened volatility, such as during the 2008 financial crisis or the COVID-19 pandemic, illustrate how market conditions can shift unpredictably, impacting investor sentiment and financial stability. High volatility is associated with increased risk and uncertainty, whereas low volatility often signals market stability. It is crucial for understanding both the risks and opportunities inherent in investments. Financial volatility can manifest in different ways and is often influenced by a combination of macroeconomic conditions, investor sentiment, and geopolitical events. In this context, volatility regimes—or distinct periods characterized by specific levels of volatility—are of particular interest to investors and risk managers. Identifying and analyzing these volatility regimes offers invaluable insights into the stability or instability of the market and helps guide investment decisions and risk management practices.

Volatility is of particular importance because it directly impacts asset pricing and risk assessment. High volatility typically indicates higher risk, as prices are more likely to swing dramatically in short timeframes. Conversely, low volatility is often associated with stable markets, where price movements are gradual and less erratic. Both scenarios are relevant for financial stakeholders, as periods of high volatility may present opportunities for profit through rapid price changes, while stable, low-volatility periods might appeal to more risk-averse investors. Therefore, understanding the dynamics of volatility is not only a tool for managing risk but also for identifying strategic investment opportunities.

However, traditional approaches to volatility analysis have limitations, especially in identifying regime shifts. Standard models, such as GARCH (Generalized Autoregressive Conditional Heteroskedasticity), primarily focus on short-term fluctuations and do not necessarily capture the broader regime changes in volatility over time. These models typically assume that volatility is driven solely by past prices or returns, overlooking the influence of external macroeconomic factors like inflation, interest rates, and economic growth. As a result, traditional models might fail to account for sudden market shifts prompted by economic crises, policy changes, or other external shocks. For instance, these models struggled to capture the sudden shifts in market dynamics during the COVID-19 pandemic, where volatility was not only driven by financial variables but also by unprecedented macroeconomic policy changes and supply chain disruptions. These models generally rely on pre-defined assumptions about state transitions, limiting their adaptability. This presents a gap in existing research and a challenge for practitioners who need to anticipate and manage volatility beyond the scope of short-term predictions.

Given these challenges, there is a pressing need to develop alternative methods for categorizing volatility regimes. Unsupervised clustering techniques offer a promising solution, as they can classify data without requiring pre-labeled inputs, making them ideal for uncovering hidden patterns in market behavior. By clustering periods of market activity based on volatility levels, we can identify distinct regimes that may align with specific economic conditions. These techniques, however, are not without challenges. Properly defining the input features and preprocessing the data are critical steps, as clustering algorithms are sensitive to initial conditions and input variables. Consequently, clustering volatility regimes requires careful consideration of both the financial data and the economic indicators that might influence regime shifts.

The problem this dissertation seeks to address is twofold: first, how to effectively categorize different volatility regimes in financial markets; and second, how to identify the macroeconomic factors that influence these regimes. Specifically, the research will explore the application of unsupervised clustering algorithms to segment periods of varying volatility in financial markets, and then analyze how economic factors such as inflation, interest rates, and GDP growth might correlate with these volatility regimes. This approach not only provides a new perspective on volatility analysis but also enhances our understanding of the broader economic context surrounding market fluctuations.

\section{Objectives of the Dissertation}

The primary objective of this dissertation is to explore the potential of unsupervised clustering algorithms in identifying volatility regimes in financial markets. Unlike traditional volatility models, which often rely on assumptions about data distribution and external shocks, unsupervised clustering does not require predefined categories or labels. This flexibility allows for the discovery of volatility regimes without relying on assumptions about market behavior, making it particularly well-suited for analyzing complex and dynamic financial data.

To achieve this objective, the dissertation will first utilize clustering algorithms to identify volatility regimes in a dataset of financial market indices. The chosen algorithms may include K-means, DBSCAN, and Gaussian Mixture Models (GMM), each of which offers unique strengths for segmenting data based on volatility patterns. K-means, for example, is well-suited for identifying distinct clusters in data with clear separation, while DBSCAN is effective for handling noise and outliers, which are common in financial datasets. GMM, with its probabilistic approach, can accommodate overlapping clusters, allowing for a more nuanced classification of volatility regimes. The use of multiple clustering algorithms will provide a comprehensive view of the data and allow for comparisons across methods, enhancing the robustness of the analysis.

Once the volatility regimes have been identified, the next objective is to analyze these regimes based on a set of macroeconomic indicators. This part of the analysis will examine how economic variables like inflation, interest rates, and GDP growth correlate with each identified regime. By mapping these indicators to different periods of volatility, the dissertation aims to uncover patterns and relationships that may help explain why certain volatility regimes arise under specific economic conditions. This analysis will involve statistical techniques to assess correlations and potential causations, providing a data-driven foundation for understanding the economic underpinnings of volatility.

A final objective of the dissertation is to explore the feasibility of using these macroeconomic factors to model or predict volatility regimes. This step will involve the development of a supervised model that uses economic indicators as input variables to classify periods according to their volatility regime. Although the primary focus of this research is on unsupervised clustering, a supervised model could add a predictive dimension, allowing for the anticipation of volatility shifts based on current economic conditions. Logistic regression or decision trees may be employed as baseline models for this purpose, as they are interpretable and well-suited for examining the relationships between input variables and categorical outcomes. This component of the research aims to provide actionable insights for practitioners, as it suggests a pathway toward predictive modeling of volatility regimes using readily available economic data.


\section{Originality of the Project}

This dissertation presents a novel approach to volatility analysis by focusing on qualitative, rather than quantitative, segmentation of volatility regimes. Unlike many traditional studies that aim to predict asset prices or forecast volatility levels directly, this research does not attempt to make explicit predictions about price movements. Instead, it seeks to understand the structural characteristics of volatility itself by categorizing different regimes and exploring their economic context. This qualitative approach aligns with the goals of risk management and strategic investment, as it emphasizes the identification of stable and unstable market periods without focusing solely on price prediction.

The originality of this project lies in its application of unsupervised learning methods to financial volatility, a domain traditionally dominated by supervised models and time-series analysis. By using clustering algorithms, this dissertation diverges from conventional volatility forecasting techniques and instead offers a fresh perspective on market analysis. Clustering allows for the discovery of hidden patterns in data, enabling a more nuanced understanding of how volatility behaves across different economic conditions. Furthermore, the inclusion of macroeconomic variables as contextual factors adds an additional layer of insight, as it allows for the examination of how external conditions influence volatility in ways that may not be captured by price-based models alone.

Additionally, this research contributes to the field by combining elements of regime-switching analysis with clustering and macroeconomic analysis. While regime-switching models, such as the Markov Switching Model, are commonly used to capture changes in market behavior, they often rely on predefined states and do not easily incorporate external factors. In contrast, the clustering approach used in this dissertation is data-driven and adaptable, making it better suited for uncovering dynamic relationships between volatility and economic conditions. By merging these two analytical perspectives, this research offers a more holistic view of volatility, with potential applications for both academic researchers and financial practitioners.

In summary, this dissertation’s originality stems from its qualitative, unsupervised approach to volatility analysis, its focus on macroeconomic context, and its integration of clustering and regime-switching concepts. This innovative approach not only broadens the scope of volatility research but also provides practical insights that could benefit investors and risk managers seeking to navigate complex market conditions. By examining volatility regimes through the lens of unsupervised clustering and macroeconomic analysis, this research aims to fill a significant gap in existing literature and contribute to a more comprehensive understanding of financial market dynamics.




