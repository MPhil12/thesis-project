\chapter*{Abstract}


Financial market volatility is a critical factor influencing investment decisions, risk management, and economic stability. Traditional volatility models, such as Generalized Autoregressive Conditional Heteroskedasticity (GARCH) and Markov Switching Models (MSM), capture some aspects of regime changes but often struggle with non-linear shifts in market dynamics. This thesis introduces an alternative approach, leveraging \textbf{unsupervised clustering techniques} to identify volatility regimes and integrating \textbf{macroeconomic factors} to enhance interpretability. We apply K-means, DBSCAN, and Gaussian Mixture Models (GMM) to categorize market periods into distinct volatility regimes and analyze their relationships with inflation, interest rates, GDP growth, and other macroeconomic indicators. The results demonstrate that unsupervised learning provides a \textbf{more flexible, data-driven framework} for understanding regime shifts. A supervised model is subsequently developed to predict volatility regimes based on macroeconomic conditions, offering insights for investors and policymakers.

